\documentclass[a4paper,12pt,oneside]{article}
%\usepackage{geometry}
%\usepackage{fancyhdr}
%\usepackage{amsmath,amsthm,amssymb }
\usepackage{graphicx}
%\usepackage{lipsum}
\usepackage[utf8]{inputenc}
%\usepackage{ngerman}
%\usepackage{parskip}
\usepackage{textcomp}
\usepackage{float}
\usepackage{hyperref}
\usepackage{placeins}

\setlength{\parskip}{0.2cm}

\hypersetup{
    pdfborder = {0 0 0},		
}

% Hurenkinder und Schusterjungen verhindern
\clubpenalty10000
\widowpenalty10000
\displaywidowpenalty=10000

\pagestyle{plain}

\title{Zeos 7.2.4 Release Notes}
\author{Jan Baumgarten}
\date{\today}
%\overfullrule=2mm
\begin{document}
\maketitle

\section{Zeos 7.2}
The Zeos Team is proud to announce the availability of Zeos 7.2.4 as a stable release.
This is the newest stable version of Zeos.
It deprecates Zeos 7.1, which will not be supported anymore.
Zeos 7.2 has seen tons of bug fixes and improvements.
We urge all people still using older versions of Zeos to upgrade.
If you have any problems with Zeos 7.2, please get in contact with us on the forums (\url{http://zeoslib.sourceforge.net}) or on the bugtracker (\url{https://sourceforge.net/p/zeoslib/tickets/}).

\section{General Changes}
\label{sec:GeneralChanges}
\subsection{Supported compilers}
\label{sec:GeneralChanges_SupportedCompilers}
Zeos 7.2 supports Delphi versions from Delphi 7 to XE 10.2 Tokyo.
Only the Win32 and Win64 compilers are supported.
Other platforms utilizing the NextGen compiler are not (yet) supported.
The Free Pascal compiler is supported from version 2.6 to version 3.0.
Older versions might work but don't get official suport by us anymore.

\subsection{Date and Time format settings}
\label{sec:GeneralChanges_DateAndTimeFormatSettings}
Zeos now supports specifying date and time format settings that will be used if Zeos doesn't know how to correctly format date and time settings for the DBMS to understand them.
This feature gets used with emulated parameters in ADO - if the FoxPro driver is used for example.
These new parameters can be set in the TZConnection.Properties property.
The format of these parameters conforms to the usual Delphi standards.
The following new parameters are supported:
\begin{itemize}
\item DateReadFormat
\item DateWriteFormat
\item DateDisplayFormat
\item TimeReadFormat
\item TimeWriteFormat
\item TimeDisplayFormat
\item DateTimeReadFormat
\item DateTimeWriteFormat
\item DateTimeDisplayFormat
\end{itemize}
The ReadFormat parameters decribe the date and time formats as the database sends them to the application.
The WriteFormat parameters describe the date and time formats as the application should send them to the database.
The DisplayFormat settings are used for strings that are supplied by the application to Zeos.

\subsection{Batch Loading}
\label{sec:GeneralChanges_BatchLoading}
We added API support for batch loading of data into databases that support this feature.
Documentation for this feature is yet to become officially available.

\section{Breaking changes}
\label{sec:BreakingChanges}
\subsection{Case sensitivity of the TZMetadata object and DBC layer meta data functions}
\label{sec:BreakingChanges_MetadataCaseSensitivity}
BEWARE: If you call Metadata functions on the DBC layer or use the TZMetadata object be sure that the case of the object name that you retrieve information for is in the correct case.
Zeos will not do any guesswork - it will simply query the underlying database for the identifier that you supply.
Example: In former versions of Zeos the call GetColumns('PEOPLE') might have returned information for the table people.
This will not happen anymore.
To query information about the table people you will have to use GetColumns('people').
If you want the former behavior restored, your call has to be like this:\\
GetColumns(\-IZConnection.\-GetMetadata.\-NormalizePatternCase('PEOPLE'))

\subsection{PostgreSQL autocommit and OID columns}
\label{sec:BreakingChanges_PostgresqlAutocommitOids}
The postgresql driver now uses the native autocommit of PostgreSQL.
Writing to OID BLOBs only works in explicit transactions because of this. 
This is a limitation in PostgreSQL and cannot be fixed in Zeos.
The proposed workaround is to use the bytea data type.
For more information see \url{https://www.postgresql.org/message-id/002701c49d7e%240f059240%24d604460a%40zaphod}.

\section{Driver specific changes}
\label{sec:DriverSpecificChanges}
\subsection{PostgreSQL}
\label{sec:DriverSpecificChanges_Postgresql}
\begin{itemize}
\item 
  The postgresql driver now uses the native autocommit of PostgreSQL. 
	OID columns can no longer be written to in autocommit mode.
	See \ref{sec:BreakingChanges_PostgresqlAutocommitOids} \nameref{sec:BreakingChanges_PostgresqlAutocommitOids}, Page \pageref{sec:BreakingChanges_PostgresqlAutocommitOids}.
\item GUID columns are now supported.
\item 
  New parameters, so PostgreSQL can connect using SSL (sslmode, requiressl, sslcompression, sslcert, sslkey, sslrootcert, sslcrl). 
	Take a look at the PostgreSQL documentation on how to use these parameters.
\item
  Zeos now can use SSPI, Kerberos and Windows Authentication with PostgreSQL.
	Just leave the username and password empty for this.
\item
  The PostgreSQL driver now maps the transaction isolation levels tiNone and tiReadUncommitted to tiReadCommitted.
	It is no longer valid to use your own transaction handling code.
	Please use the TZConnection methods StartTransaction, Commit and Rollback.
\item
  The PostgreSQL driver now supports read only transactions.
\item
  The PostgreSQL driver now supports +Infinity, -Infinity and NaN for floating point values.
\item
  If you still use PostgreSQL 7 databases, we urge you to move on to a newer version.
	PostgreSQL 7 is deprecated with this version and will be removed with Zeos 7.3.
\end{itemize}

\subsection{Firebird / Interbase}
\label{sec:DriverSpecificChanges_FirebirdInterbase}
\begin{itemize}
\item we added support for Firebird 3.0.
\item new parameter to enable Firebird 3 wire compression: wirecompression
\item support for Firebird 3.0 boolean fields
\item we reenabled the use of Firebird and Interbase autocommit
\item
  We added support for the new DBC layer batch loading API to the Firebird / Interbase driver.
	For more information see \ref{sec:GeneralChanges_BatchLoading} \nameref{sec:GeneralChanges_BatchLoading}, Page \pageref{sec:GeneralChanges_BatchLoading}.
\item 
  If you still use Interbase 5 databases, we urge you to move on to a newer version.
	Interbase 5 is deprecated with this version and will be removed with Zeos 7.3.	
\end{itemize}

\subsection{MySQL / MariaDB}
\label{sec:DriverSpecificChanges_MysqlMariadb}
\begin{itemize}
\item
  TZQuery and TZReadOnlyQuery now support the use of multiple statements in the query.
	The first result that is returned by the server will be the result that gets loaded into the Dataset.
\item
  The MySQL driver should now be thread safe.
	This still means that threads are not allowed to share a connection.
\item
  New connection level parameter MySQL\_FieldType\_Bit\_1\_IsBoolean.
	If this parameter is enabled, fields declared as BIT(1) will be treated as boolean fields.
	The old assumption that an enum('Y','N') is a boolean field is disabled if this parameter is enabled.
	If this parameter is enabled, enum('Y','N') will be a string field.
	Other enums behave as before, they will be mapped to a sting filed in any case.
	This parameter will be enabled by default in Zeos 7.3.
\end{itemize}

\subsection{MS SQL / SAP ASE (Sybase ASE)}
\label{sec:DriverSpecificChanges_MssqlAse}
\begin{itemize}
\item 
  reenabled Sybase support.
	This should allow some basic usage, but your mileage may vary.
	If you have problems please get in contact.
\item the driver now supports GUID-Columns on MS SQL Server.
\item enable support of TDS 5.0 for newer Sybase Servers
\item with FreeTDS the server port can now be specified
\item 
  The FreeTDS drivers now use sybdb.dll as the default dll to load.
	Watch out if your program doesn't set the LibraryPath property in TZConnection.
\item 
  If you still use the mssql protocol using ntwdblib.dll to connect to MS SQL Server, we urge you to move on to either use ADO or FreeTDS.
	The ntwdblib.dll is not supported by Microsoft for ages now and we will discontinue its support with Zeos 7.3.
\end{itemize}

\subsection{Oracle}
\label{sec:DriverSpecificChanges_Oracle}
\begin{itemize}
\item
  Performance improvement: The oracle driver now supports block cursor reads.
	This allows to fetch more than one record in one network roundtrip.
	The parameter for setting the block size is internal\_buffer\_size.
\item
  We added support for the new DBC layer batch loading API to the Oracle driver.
	For more information see \ref{sec:GeneralChanges_BatchLoading} \nameref{sec:GeneralChanges_BatchLoading}, Page \pageref{sec:GeneralChanges_BatchLoading}.
\end{itemize}

\subsection{ADO}
\label{sec:DriverSpecificChanges_Ado}
\begin{itemize}
\item
  The ADO driver now also supports Free Pascal.
\item
  We added support for the new DBC layer batch loading API to the ADO driver.
	For more information see \ref{sec:GeneralChanges_BatchLoading} \nameref{sec:GeneralChanges_BatchLoading}, Page \pageref{sec:GeneralChanges_BatchLoading}.
\item
  Zeos emulates named parameters for ADO drivers that don't support parameters.
	Unfortunately this means Zeos doesn't know how to correctly format timestamps and similar data types to be correctly recognized by the underlying database.
	Please use the new connection parameters ReadFormatSettings, WriteFormatSettings, DisplayFormatSettings.
\end{itemize}

\section{Known Problems}
\label{sec:KnownProblems}
\begin{itemize}
\item 
  Zeos currently doesn't support the BCD type columns of Delphi.
	NUMERIC and DECIMAL columns still get mapped to floating point types.
	This will be adressed in Zeos 7.3 because it requires a lot of changes in the Zeos core.
\end{itemize}

\section{The future development of Zeos}
\label{sec:FutureDevelopmentOfZeos}
The next version of Zeos will be Zeos 7.3 which currently is in the making.
Currently the following changes and features are planned:
\begin{itemize}
\item support for OLEDB
\item support for ODBC
\item support for BCD type columns to allow correct usage of NUMERIC and DECIMAL fields
\item support for GUID type columns in Firebird
\item Interbase 5 will not be supported officially anymore
\item PostgreSQL 7 databases will not be supported officially anymore
\item the use of ntwdblib.dll will not be supported anymore.
\item DBC layer: the use of TZDriverManager.Connect will not be supported with a string anymore, only the use of a TZURL object will be supported.
\item
  Drivers with version numbers will not be used in Zeos 7.3 anymore (i.e. firebird-2.5 will become firebird).
  Please migrate.
\end{itemize}

\end{document}