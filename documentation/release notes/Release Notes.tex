\documentclass[a4paper,12pt,oneside]{article}
%\usepackage{geometry}
%\usepackage{fancyhdr}
%\usepackage{amsmath,amsthm,amssymb }
\usepackage{graphicx}
%\usepackage{lipsum}
\usepackage[utf8]{inputenc}
%\usepackage{ngerman}
%\usepackage{parskip}
\usepackage{textcomp}
\usepackage{float}
\usepackage{hyperref}
\usepackage{placeins}

\setlength{\parskip}{0.2cm}

\hypersetup{
    pdfborder = {0 0 0},		
}

% Hurenkinder und Schusterjungen verhindern
\clubpenalty10000
\widowpenalty10000
\displaywidowpenalty=10000

\pagestyle{plain}

\title{Zeos 8.0 Release Notes}
\author{Jan Baumgarten}
\date{\today}
%\overfullrule=2mm
\begin{document}
\maketitle

\section{Zeos 8.0}
The Zeos Team is proud to announce the availability of Zeos 8.0 as a stable release.
This is the newest stable version of Zeos.
It deprecates Zeos 7.2 and all prior versions of Zeos.
Zeos 8.0 has seen tons of bug fixes and improvements.
We urge all people still using older versions of Zeos to upgrade.
If you have any problems with Zeos 8.0, please get in contact with us on the forums (\url{http://zeoslib.sourceforge.net}) or on the bugtracker (\url{https://sourceforge.net/p/zeoslib/tickets/}).

\section{General Changes}
\label{sec:GeneralChanges}
\subsection{Supported compilers}
\label{sec:GeneralChanges_SupportedCompilers}
Zeos 8.0 supports Delphi versions from Delphi 7 to XE 10.4 Seattle.
Besides the x86 and x86\_64 compilers, Zeos now also supports the Nextgen compilers and so supports all available platforms on Delphi.
So on Delphi Windows, macOS / Mac OS X, iOS and Android are supported now.
The Free Pascal compiler is supported from version 3.0 to version 3.2.
Older versions might work but don't get official support by us anymore.

\subsection{New Drivers}
\label{sec:GeneralChanges_NewDrivers}
Zeos 8.0 now comes with some new drivers:
\begin{itemize}
  \item OleDB
	\item ODBC including UnixODBC
	\item Firebird with new interface based API
	\item Webservice Proxy Driver
\end{itemize}
For a description of these new drivers please see the respective chapters.

\subsection{no more driver versioning}
We deprecated protocols with server versions during the life time of Zeos 7.2 already.
These protocols have been removed from Zeos 8.0.
So protocol names like "firebird-3.0" will not work anymore.
Please use protocol names without version numbers instead, like "firebird".

\subsection{Property Documentation}
Zeos drivers support a multitude of parameters. via the "`Properties"' property.
These parameters are now collected in the new file ZDbcProperties.pas and also in ZPropertiesEditor.pas.
Based on the latter file, we will release a more detailed documentation for Zeos driver properties.

\subsection{Support for connection loss}
Zeos 8.0 now supports the programmer in reacting to a connection loss.

\textbf{Jan:} How does this work and which drivers are supported Egonhugeist?

\subsection{New prepare logic}
Some drivers don't prepare statements on first execution automatically.
Instead they prepare statements after some executions now.
This is for cases where statements get used once only.
Preparing these statements would be a waste of resources in that case.
This behavior can be influenced by the user.

\textbf{Jan:} Ask Egonhugeist how this is influenced and which drivers work this way.

\subsection{tokens don't have local string variables anymore}
Tokens don't have local string variables anymore.
They are only a PChar and a Length.
This reducees memory impact and keeps the memory manager out of things.
Tokens are kept in a TZTokenList.
This List should be used as an interface by users.
It will copy strings as needed.
One shouldn't modify anything in the PChars because this will lead to a change in the SQL statement of the user.

\subsection{Discontinued support for character agnostic connection character sets}
It is hard work and a big impact on Zeos to have support for connection agnostic character sets.
In the past Zeos tried to read the metadata of the database and to convert the incoming characters accordingly to the current application character set.
Since database systems do exactly the same if they know the correct character set, we effectively try to double efforts where databases already are good.
With Zeos 8.0 we will cancel these efforts.
This greatly simplifies Zeos and leaves room for Zeos developers to work on other things.
Starting with Zeos 8.0 we raise exceptions if someone tries to use the following character sets on the specified databases as a connection character set:

\begin{itemize}
  \item 
    PostgreSQL: SQL\_ASCII
  \item
		MySQL: BINARY
	\item
    Firebird: none, octetts
\end{itemize}

\subsection{DBC layer: changed GetPWideChar / GetPAnsiChar Len parameter}
The Len parameter of the GetPWideChar and GetPAnsiChar method is no longer a pointer.
A var parameter is used instead.

\subsection{Support for TBCDField / TFMTBCDField}
For correct support of NUMERIC and DECIMAL field types, Zeos now supports the use of TBCDField and TFMTBCDField.
The DBC layer has been extended accordingly.
If you use static field definitions in your forms, you will need to adapt them before compiling your application for Zeos 8.0.

\textbf{Note:}
TBCDField doesn't usually round values.
Assigning 1.065 works for a numeric(15,2) database field.
This value usually will also be presented to the user (TField.AsString).
To remedy situations where the database would round the value and Zeos and the database have a different view of the fields value, Zeos will round assigned values to the scale defined by the underlying database fields before writing them to the field buffer.
So Zeos will round 1.065 to 1.07 and then send 1.07 to the database.

\subsection{doPreferPrepared defaults to on now}
The option doPreferPrepared defaults to on now.
This leads to Zeos preparing all statements automatically if it sees fit to do so.
For more information see the documentation on this parameter.

\textbf{Jan:}@Egonhugeist: Is this correct?

\subsection{Ordinal position of return value for stored procedures}
For database systems, where a return value is defined, like MS SQL Server, the return value has moved to ordinal position 0.
Please check your applications.

\subsection{support for aborting long running operations}
Zeos now supports the abortion of long running operations.
To do so, call TZConnection.AbortOperation from another thread.
This is supported on MySQL, MariaDB, Oracle, dblib (MS SQL Server, SAP Adaptive Server Enterprise), PostgreSQL, Firebird and SQLite.

\subsection{new method StartTransaction on the DBC layer}
Until Zeos 8.0 the DBC layer didn't have a way to start transactions besides leaving AutoCommit mode.
Zoes 8.0 introduces a new StartTransaction method.
This can save API calls - especially on databases that default to using auto commit because it doesn't start unnecessary transactions after Commit / Rollback.

\subsection{Nested transactions}
Zeos now supports nested transactions. TZConnection.StartTransaction on the component layer and IZConnection.StartTransaction will return the current transaction nesting level.
If a transaction already has been started, Zeos will use save points to create nested transactions automatically.
When issuing a Commit or Rollback, Zeos will act accordingly by releasing save points or rolling back to save points.

\subsection{stBigDecimal and stCurrency have new meaning}
The two data types stBigDecimal and stCurrency are not used for describing TExtendedFields.
They now are used for describing decimal and numeric field types.

\subsection{Zeos Field Classes}
Zeos introduced its own field classes.
They derive from the original Fields, so can be used almost seamlessly.
The Zeos field classes can work directly on Zeos internal structures and by this way are way faster then normal TField descandants.
Also Zeos implements some field classes to support features of databases, that are not supported by the Delphi / Free Pascal TField implementations.

\textbf{Jan:}Option to disable this behavior?

\subsection{New parameters for handling of exceptions and warnings}
Zeos 8.0 introduces two new parameters for the handling of exceptions and warnings:

\subsubsection{AddLogMsgToExceptionOrWarningMsg }
This is a boolean option.
If set to true (default) The offending SQL will be added to exceptions generated by the server.
If set to false, SQL statements will not be added.

On the DBC layer this option is reflected by the SetAddLogMsgToExceptionOrWarningMsg method of the IZConnection Interface.

\subsubsection{RaiseWarningMessages}
This is a boolean option.
If set to true, warnings by the SQL server will be raised as exceptions.
Default is false.

On the DBC layer this option is reflected by the SetRaiseWarningMessages method of the IZConnection Interface.

\subsection{updates on more / joined queries}
Until Zeos 7.2 Zeos could update queries only if they queried a single table and all fields were writable.
Starting with Zeos 8.0 Zeos uses information from ProviderFlags and TField.ReadOnly.
This allows us to have more queries be updateable automatically.
See [\#310] and [r6882].

\textbf{Jan:}We need more information on how these things work for the information to be more useful.

\subsection{output parameter support on TZQuery}
EH@Jan Outparamter support added for the TZQuery.
Means all SP's can be executed without using the TZStoredProc component.
So the TZStoredProc acts like TZTable and is just a name wrapper.
IZCallableStatement just expects the SP name.
To get out-params running the parameter IO needs to be set explicit and for some drivers the sizes of variable types need to be set too.

\textbf{Jan:}We need more information on how these things work for the information to be more useful.

\subsection{smaller memory footprint}
By optimizing access in TDataset descandants, Zeos is able to save ion memory now if TZFields are used, which is the default for Zeos 7.3.

\textbf{Jan:}Is this correct? Regarding dependency on TZFields?

\subsection{Binding technique for parameters has changed}

Binding technic of parameters (ZDBC) has been changed to "Immediate" (Jan: Early?) bindings.
Late bindings are supported in case the paramaters can't be described once or if the driver (SQLite) doesn't allow to read values back i.e. Logging mechanism.
Those precaching of the values have been changed away from the slow TZVariants and using a memory optimised TZBindList instead.

\textbf{Jan:}What is the impact on the end user?

\subsection{New field types for older compilers}
New fields added.
Some compilers still do not support all field types ZDBC is able to distinguish.
That has been improved now.
The more we've noticed an enormous performance problem of the generic fields caused by GetFieldData / SetFieldData moves.
The new fields a loads faster now.

\textbf{Jan:}This should be incorporated into one single chapter about TZFields.

\subsection{TZParam}

TZParam will follow up. 
Not yet commited.

\textbf{Jan:}I assume this is important for explaining how batch bindings work on Zeos?

\subsection{TZTransaction}

\textbf{Jan:}Needs documentation. Has to be derived from \url{https://zeoslib.sourceforge.io/viewtopic.php?f=50&p=159710#p159710}

\section{Firebird}

\subsection{Support for TGUID-Field}
Zeos now supports different ways to store GUIDs in Firebird and map them to TGUIDField:
\begin{itemize}
  \item SetGUIDByType -> all CHAR(16) CHARACTER SET OCTETS type columns will be treated as GUID columns. Can be used on Connection and DataSet.
  \item GUIDDomains -> list of database domains that are to be treated as GUID columns. List separators are ';' and ','. Can be used on connection only.
  \item GUIDFields -> list of fields that should be treated as GUID type fields. List separators are ';' and ','. Can be used on Dataset only.
\end{itemize}

\subsection{Support for timeouts on Firebird 4.0}
Zeos now supports more timeouts, as introduced with Firebird 4.0:
\begin{itemize}
  \item StatementTimeOut
  \item SesssionIdleTimeOut
\end{itemize}


\end{document}